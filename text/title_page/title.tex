\noindent\begin{minipage}{\textwidth}
\begin{center}
\thispagestyle{empty}
\vspace{0.5cm}
{ \Large{Peculiar features of missing mass distributions in studies of exclusive reactions}}\\
\vspace{1cm}

{\large Iu.A. Skorodumina$^{1, a}$, G.V. Fedotov$^{2}$,  R.W. Gothe$^{1}$} \\[16pt]

\parbox{.86\textwidth}{\centering\footnotesize\it
$^1$Department of Physics and Astronomy, University of South Carolina, Columbia, SC\\[8pt]
\setstretch{0.3} 
$^2$National Research Centre ``Kurchatov Institute" B. P. Konstantinov Petersburg Nuclear Physics Institute, Gatchina, St. Petersburg, Russia\\
[20pt]
E-mail: $^a$skorodum@jlab.org}\\


\vspace{2cm}
{\bf Abstract}\\[9pt]

\end{center}
\everypar{\looseness=-1}
{\small This study examines the influence of the following five factors on missing mass distributions, radiative effects, admixtures from other channels, detector resolution, Fermi smearing, and final state interactions with spectator nucleons. All factors are considered independently of each other. The examination is supported by theoretical calculations and exemplified by distributions of the quantities $M^{2}_{X[0]}$ and $M^{2}_{X[\pi^{-}]}$ that are produced by the Monte Carlo simulation for the double-pion electroproduction off protons. The study also includes a naive modeling of kinematic effects of final state interactions with spectator nucleons.}


\end{minipage}

