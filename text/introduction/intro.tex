\newpage
\chapter{Introduction}
\mbox{}\vspace{-\baselineskip}


Nowadays experimental investigations of exclusive meson photo- and electroproduction off protons have been put into high gear worldwide. When performing such an investigation, it is very important to properly isolate events that correspond to a particular reaction channel from the bulk of experimental data. This challenging task is commonly resolved employing the ``missing mass" technique.

During the data analysis, events of a desired exclusive channel are selected from the experimentally available event sample, which typically corresponds to the process $ep \rightarrow e'HX$, where $H$ denotes all registered final hadrons, while $X$ stands for all unregistered particles\footnote[1]{The scattered electron $e'$ must be registered here as it defines the reaction event.}. Then, a missing mass $M_{X}$ is a quantity that is calculated via energy-momentum conservation from the four-momenta of the registered particles ($e'$, $H$) and its distribution reflects the mass spectrum of the unregistered part ($X$). The examination of missing mass distributions then gives an opportunity to qualify the presence of any type of background in the investigated event sample and to judge the reliability of the entire event selection.


Although in general a missing mass of any number of unregistered particles can be considered, usually close attention is paid to the following two quantities, the missing mass squared of one missing hadron $h$ ($M^{2}_{X[h]}$) and the missing mass squared of the fully exclusive reaction ($M^{2}_{X[0]}$). These quantities serve the purpose of isolating the exclusive events best, as they form a discrete spectrum in the absence of such additional factors as detector resolution, radiative effects, background admixtures, etc. 

Employing the missing mass technique in the analysis allows using event samples with one unregistered final hadron along with the fully exclusive event sample, distinguishing in this way between so-called reaction ``topologies", each of which corresponds to a particular combination of registered final hadrons.  The number of available reaction topologies is then equal to $n+1$, where $n$ is the number of hadrons in the reaction final state. This approach allows an increase of the analyzed statistics (which is sometimes significant).

In order to pick out the exclusive reaction, in each topology the missing mass distribution is subject to a so-called ``exclusivity cut'' as a final step of the event selection. A properly chosen position of the exclusivity cut allows at least the suppression of the background and non-physical admixtures or even their complete removal.


Meanwhile, experimentally obtained missing mass distributions are inevitably folded with the influence of various factors such as detector resolution, radiative effects, background admixtures, etc. These factors disturb the missing mass distribution by changing its shape, width, and maximum position, hence complicating the task of isolating the desired exclusive channel. Therefore, to better understand the combined influence of these factors, it is extremely important to understand the impact of each factor by itself.


\everypar{\looseness=-1}
This study examines the influence of the following five factors on the missing mass distributions, i.e. radiative effects, admixture from other channels, detector resolution, Fermi smearing, and final state interactions with spectator nucleons. All factors are considered independently of each other. The examination is supported by theoretical calculations and exemplified by distributions of the quantities $M^{2}_{X[\pi^{-}]}$ and $M^{2}_{X[0]}$ that are produced by the Monte Carlo simulation for the reaction of double-pion electroproduction off protons. All conclusions, however, can be simply generalized for any missing particle and any exclusive reaction.


The Monte-Carlo simulation, which was performed to visually illustrate the manifestation of each factor, employs a particular well-established technique (see details in Sect.~3.1-3.4) for all considered factors except the last one, i.e. final state interactions with spectator nucleons. Unfortunately, no methods currently exist to simulate the latter. Therefore, to trace the impact of this factor on the missing quantities, a naive modeling of kinematic features of final state interactions with spectator nucleons is attempted in Sect.~\ref{sec:fsi}.


All histograms in this note are plotted under the following conditions, $E_{beam}$ = 2.039~GeV, 1.4~GeV $< W <$ 1.8~GeV and 0.4~GeV$^{2}$ $< Q^{2} <$ 0.6~GeV$^{2}$, and (unless specified otherwise) filled with events generated by the TWOPEG~\cite{twopeg} event generator. All distributions are normalized in a way that the maxima of the main peaks are equal to one.


To avoid confusion, the reader is strongly encouraged to pay close attention to the definition of the examined missing quantities, which is given in Chapter~\ref{ch:mm}, before considering any conclusions made in this study. 


